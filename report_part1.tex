\documentclass[11pt]{article}
\usepackage[margin=1in]{geometry}
\usepackage{graphicx}
\usepackage{booktabs}
\usepackage{hyperref}
\usepackage{amsmath}

\title{ESE 3060 Final Project -- Part 1\\
\large Effect of Learning Rate Warmup Ratio on CIFAR-10 Training Speed}
\author{[Your Name(s)]}
\date{December 8, 2025}

\begin{document}
\maketitle

\section{Hypothesis}

We hypothesize that the default learning rate warmup ratio of 23\% in the airbench94 training script is suboptimal. During warmup, the learning rate linearly increases from 20\% to 100\% of the target value. If this warmup period is too long, training steps are ``wasted'' at low learning rates, potentially slowing convergence.

Conversely, if warmup is too short, the optimizer may become unstable due to large gradient updates early in training when weights are randomly initialized. We test warmup ratios ranging from 5\% to 35\% to find the optimal balance between stability and training efficiency.

\section{Method}

\subsection{Experimental Setup}
\begin{itemize}
    \item \textbf{Baseline}: airbench94.py with default \texttt{warmup\_ratio = 0.23}
    \item \textbf{Hardware}: [Specify your GPU, e.g., NVIDIA A100]
    \item \textbf{Runs per variant}: 25 (for statistical significance)
    \item \textbf{Metric}: Test accuracy with test-time augmentation (TTA)
\end{itemize}

\subsection{Warmup Ratios Tested}
\begin{center}
\begin{tabular}{cc}
\toprule
Variant & Warmup Ratio \\
\midrule
V1 & 0.05 (5\%) \\
V2 & 0.10 (10\%) \\
V3 & 0.15 (15\%) \\
V4 & 0.20 (20\%) \\
\textbf{Baseline} & \textbf{0.23 (23\%)} \\
V5 & 0.30 (30\%) \\
V6 & 0.35 (35\%) \\
\bottomrule
\end{tabular}
\end{center}

\subsection{Code Changes}
We modified line 383 of \texttt{airbench94.py} to use a configurable warmup ratio:
\begin{verbatim}
warmup_ratio = hyp['opt']['warmup_ratio']  # Now configurable
warmup_steps = int(total_train_steps * warmup_ratio)
\end{verbatim}

\section{Results}

% TODO: Replace with your actual results
\begin{table}[h]
\centering
\caption{Experimental Results}
\begin{tabular}{ccccc}
\toprule
Warmup Ratio & Mean Acc (\%) & Std (\%) & Mean Time (s) & Significant? \\
\midrule
0.05 & XX.XX & X.XX & X.XX & -- \\
0.10 & XX.XX & X.XX & X.XX & -- \\
0.15 & XX.XX & X.XX & X.XX & -- \\
0.20 & XX.XX & X.XX & X.XX & -- \\
\textbf{0.23} & \textbf{XX.XX} & \textbf{X.XX} & \textbf{X.XX} & baseline \\
0.30 & XX.XX & X.XX & X.XX & -- \\
0.35 & XX.XX & X.XX & X.XX & -- \\
\bottomrule
\end{tabular}
\end{table}

% TODO: Uncomment and update path after generating figures
% \begin{figure}[h]
% \centering
% \includegraphics[width=0.9\textwidth]{figures/combined_results.pdf}
% \caption{Effect of warmup ratio on accuracy (left) and training time (right). Orange bar indicates baseline.}
% \end{figure}

\section{Conclusion}

% TODO: Fill in based on your results
[Summarize your findings here. Did reducing warmup help? Did any variant significantly outperform baseline? What does this tell us about the importance of warmup in this training setup?]

\textbf{Key findings:}
\begin{itemize}
    \item [Finding 1]
    \item [Finding 2]
    \item [Finding 3]
\end{itemize}

\section{Experiment Logs}
Full experiment logs are available in the \texttt{logs/} directory of the submitted repository. The analysis script \texttt{analyze\_results.py} reproduces all tables and figures.

\end{document}
